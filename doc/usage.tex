\begin{paragraph}
Using MarTeX for your website shenanigans is pretty easy. You start by requiring martex.php and construction a MarTeX instance. All MarTeX classes and functions are contained in the MarTeX namespace, so there won't be any collissions with any functions and classes you use yourself.

\begin{code}{php}
require_once (__DIR__.'/relative/path/to/martex.php');
    
\$Tex = new MarTeX\backslashMarTeX();
\end{code}

Then, you require the modules you like and register them to the MarTeX instance.

\begin{code}{php}
require_once (__DIR__.'/relative/path/to/document.php');

\$Document = new MarTeX\backslashDocument();
\$Tex-\>registerModule(\$Document);
\end{code}

Now you are all set up for TeX'in. Use the parse function with some text. This could be a file, or POST data. If you use POST data, 
make sure it is urldecoded and magic quotes are turned off in your php configuration. Otherwise your backslashes in your text could end up being escaped, 
and MarTeX will interpret them as a double backslash, resulting in a plain old newline.

\begin{code}{php}
\$Tex-\>parse(\$my_tex_text);
\end{code}

We're almost finished. The last thing to do is to check for errors and maybe return them, and returning the output.

\begin{code}{php}
if (\$Tex-\>hasError()) {
    echo \$Tex-\>getError();
}
echo \$Tex-\>getResult();
\end{code}

Please note that, even though we might have errors, the result is still returned. In most cases this works, 
since MarTeX will usually just discard commands that cause errors, or make an attempt to fix it. Only on
serious syntax errors MarTeX will not return any output.
\end{paragraph}
